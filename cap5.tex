%%%%%%%%%%%%%%%%%%%%%%%%%%%%%%%%%%%%%%%%%%%%%%%%%%%%%%%%%%%%%%%%%%%%%%%%%%%%%
% Chapter 5: Conclusiones y Trabajos Futuros 
%%%%%%%%%%%%%%%%%%%%%%%%%%%%%%%%%%%%%%%%%%%%%%%%%%%%%%%%%%%%%%%%%%%%%%%%%%%%%%%

%++++++++++++++++++++++++++++++++++++++++++++++++++++++++++++++++++++++++++++++

CodeLab nace como una herramienta que pretende extender las funcionalidades
de otras herramientas como Github Classroom, incorporando respuestas a las necesidades 
específicas del profesorado para apoyar la gestión de los cursos y
también la corrección de prácticas.

El uso de la plataforma se ha diseñado pensando en su facilidad de
uso para aquellos que no estén tan familiarizados con los conceptos
de GitHub. 

Por otro lado, se ha tenido especial cuidado en cumplir lo máximo
posible con los estándares del Software Libre,
facilitando el desarrollo colaborativo y facilitando 
la extensión de nuevas funcionalidades por parte de otros programadores.

Creemos que las facilidades que proporciona CodeLab son
útiles para aquellos profesores que tengan grupo grandes de
alumnos incluso si no están familiarizados con el entorno de Github.
CodeLab automatiza la creación de repositorios y el manejo
de los permisos de acceso de cada alumno a sus repositorios y a los
repositorios de los compañeros

Además, el control de versiones ofrece muchas ventajas a los
desarrolladores. En la actualidad, todas las empresas de desarrollo
usan control de versiones git, por lo que, es fundamental que los
alumnos aprendan a manejar git de forma correcta y aprendan técnicas
de trabajo en equipo así como todas las herramientas que proporciona
Github como los issues o los projects.

En el futuro, me planteo seguir desarrollando CodeLab, mejorándo la herramienta
y añadiendole nuevas funcionalidades. 
Una de las primeras mejoras
que se plantean es la utilización de una librería de front-end como
Vue o React para mejorar la calidad visual de la plataforma web.
Otra nueva funcionalidad que me gustaría añadir es la posibilidad
de más de un profesor por aula.
